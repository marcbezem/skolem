%\documentclass[12pt,a4paper]{amsart}
\documentclass[10pt,a4paper]{article}
\ifx\pdfpageheight\undefined\PassOptionsToPackage{dvips}{graphicx}\else%
\PassOptionsToPackage{pdftex}{graphicx}
\PassOptionsToPackage{pdftex}{color}
\fi

\usepackage{diagrams1}
\usepackage[all]{xy}
\usepackage{url}
\usepackage{verbatim}
\usepackage{latexsym}
\usepackage{amssymb,amstext,amsmath}

\usepackage{float}
\floatstyle{boxed}
\restylefloat{figure}


%%%%%%%%%
\def\oge{\leavevmode\raise
.3ex\hbox{$\scriptscriptstyle\langle\!\langle\,$}}
\def\feg{\leavevmode\raise
.3ex\hbox{$\scriptscriptstyle\,\rangle\!\rangle$}}

%%%%%%%%%

\newcommand\myfrac[2]{
 \begin{array}{c}
 #1 \\
 \hline \hline 
 #2
\end{array}}

\newcommand*{\Scale}[2][4]{\scalebox{#1}{$#2$}}%
\newcommand*{\Resize}[2]{\resizebox{#1}{!}{$#2$}}

\newcommand{\ideal}[1]{\langle #1\rangle}
\newcommand{\elim}{{\sf elim}}
\newcommand{\dM}{{\sf dM}}
\newcommand{\Cir}{\mathsf{S^1}}
\newcommand{\base}{{\sf base}}
\newcommand{\inc}{{\sf inc}}
\newcommand{\inh}{\mathsf{inh}}
\newcommand{\squash}{{\sf squash}}
\newcommand{\transp}{{\sf transp}}
\newcommand{\squeeze}{{\sf squeeze}}
\newcommand{\LOOP}{{\sf loop}}
\newcommand{\Top}{{\sf Top}}
\newcommand{\Sys}{{\sf S}}
\newcommand{\Ref}{{\sf Ref}}
\newcommand{\LINE}{{\sf line}}
\newcommand{\JJ}{{\sf J}}
\newcommand{\II}{\mathbb{I}}
\newcommand{\PP}{{\sf Path}}
\newcommand{\Sp}{{\sf S}}
\newcommand{\TT}{\mathbb{F}}
\newcommand{\FF}{\mathbb{F}}
\newcommand{\DD}{\mathbb{D}}
\newcommand{\Iso}{{\sf Iso}}
\newcommand{\refl}{{\bf refl}}
\newcommand{\HH}{{\sf H}}


\newcommand{\mkbox}[1]{\ensuremath{#1}}


\newcommand{\Id}{{\sf Id}}
\newcommand{\ident}{{\sf id}}
\newcommand{\Path}{{\sf Path}}
\newcommand{\IdP}{{\sf IdP}}
\newcommand{\ID}{{\sf ID}}
\newcommand{\Equiv}{{\sf Equiv}}
\newcommand{\isEquiv}{{\sf isEquiv}}
\newcommand{\ext}{\mathsf{ext}}
\newcommand{\lift}{\mathsf{lift}}
\newcommand{\cof}{\mathsf{cof}}
\newcommand{\isContr}{{\sf isContr}}
\newcommand{\Fib}{{\sf Fib}}

\newcommand{\CC}{{\mathcal C}}
\newcommand{\subst}{{\sf subst}}
\newcommand{\res}{{\sf res}}
\newcommand{\Int}{{\bf I}}
\newcommand{\sem}[1]{\langle #1\rangle}

\newcommand{\Sph}{{\sf S}^1}
\newcommand{\PROP}{{\sf prop}}
\newcommand{\SET}{{\sf set}}
\newcommand{\pair}[1]{{\langle #1 \rangle}}
\newcommand{\openb}[1]{\mathsf{b}(#1)}
%\newcommand{\pth}[1]{\mathsf{P}(#1)}
\newcommand{\pth}[1]{#1^{\II}}
%\newcommand{\cyl}[1]{\mathsf{Cyl}(#1)}
\newcommand{\cyl}[1]{#1\times\II}
\newcommand{\free}[1]{\mathsf{D}(#1)}
\newcommand{\Prod}[2]{\displaystyle\prod _{#1}~#2}
\newcommand{\Sum}[2]{\displaystyle\sum _{#1}~#2}
\newcommand{\gothic}{\mathfrak}
\newcommand{\omicron}{*}
\newcommand{\gP}{{\gothic p}}
%\newcommand{\lift}[1]{\tilde{#1}}
\newcommand{\gM}{{\gothic M}}
\newcommand{\gN}{{\gothic N}}
\newcommand{\rats}{\mathbb{Q}}
\newcommand{\ints}{\mathbb{Z}}

\usepackage{epsf}
\usepackage{epsfig}
% \usepackage{isolatin1}
\usepackage{a4wide}
\usepackage{verbatim}
\usepackage{proof}
\usepackage{latexsym}
\usepackage{amssymb}
% \usepackage{stmaryrd}
\newcommand{\abs}[2]{\lambda #1 . #2}            % abstraction of #1 in #2
\usepackage{mytheorems}
%\newtheorem{proposition}[theorem]{Proposition}

%\documentstyle{article}
\newcommand{\IF}[3]{{{\sf if}~#1~{\sf then}~#2~{\sf else}~#3}}
\newcommand{\lfpi}[3]{(\Pi #1{:}#2)#3}
\newcommand{\HA}{{\sf HA}}
\newcommand{\AC}{{\sf AC}}
\newcommand{\HAw}{\hbox{\sf{HA}$^{\omega}$}}
\newcommand{\EM}{\hbox{\sf{EM}}}
\newcommand{\DC}{\hbox{\sf{DC}}}
\newcommand{\BB}{\hbox{\sf{B}}}

\def\NN{\hbox{\sf N}}
\def\Type{\hbox{\sf Type}}
%\def\Box[1]{\mathsf{b}(#1)}
\def\PER{\hbox{\sf PER}}
\def\FUN{\Pi}
\def\ELEM{\hbox{\sf El}}
\def\GG{\hbox{\sf G}}
\def\TP{\hbox{\sf TP}}
\def\N0{\hbox{\sf N}_0}
\def\ZERO{\hbox{\sf zero}}
\def\SUCC{\hbox{s}}
\setlength{\oddsidemargin}{0in} % so, left margin is 1in
\setlength{\textwidth}{6.27in} % so, right margin is 1in
\setlength{\topmargin}{0in} % so, top margin is 1in
\setlength{\headheight}{0in}
\setlength{\headsep}{0in}
\setlength{\textheight}{9.19in} % so, foot margin is 1.5in
\setlength{\footskip}{.8in}

% Definition of \placetitle
% Want to do an alternative which takes arguments
% for the names, authors etc.

\newcommand{\lbr}{\lbrack\!\lbrack}
\newcommand{\rbr}{\rbrack\!\rbrack}
%\newcommand{\sem}[2] {\lbr #1 \rbr_{#2}}  % interpretation of the terms
\newcommand{\PAIR}[2] {{<}#1,#2{>}}  % interpretation of the terms
\newcommand{\add}{\mathsf{add}}
\newcommand{\app}{\mathsf{app}}
\newcommand{\APP}{\mathsf{APP}}
\newcommand{\BAPP}[2]{\mathsf{app}(#1,#2)}
\newcommand{\nat}{{N}}
\newcommand{\NNO}{\hbox{\sf N$_0$}}
\newcommand{\UU}{\hbox{\sf U}}
\newcommand{\VV}{\hbox{\sf V}}
\newcommand{\EXIT}{\mathsf{exit}}
\newcommand{\natrec}{\hbox{\sf{natrec}}}
\newcommand{\boolrec}{\hbox{\sf{boolrec}}}
\newcommand{\nil}{[]}
\newcommand{\cons}{\mathsf{cons}}
\newcommand{\lists}{\mathsf{list}}
\newcommand{\VEC}{\mathsf{vec}}
\newcommand{\reclist}{\mathsf{RecL}}
\newcommand{\vect}{\mathsf{vect}}
\newcommand{\brecp}{\Psi}
\newcommand{\true}{\mathsf{true}}
\newcommand{\false}{\mathsf{false}}
\newcommand{\bool}{{N_2}}
\newcommand{\ifte}[3]{\mathsf{if}\ #1\ \mathsf{then}\ #2\ \mathsf{else}\ #3}
\newcommand{\nats}{\mathbb{N}}
\newcommand{\Con}{{\sf Con}}
\newcommand{\Typ}{{\sf Type}}
\newcommand{\Elem}{{\sf Elem}}
\newcommand{\Char}{{\sf Char}}
%\newcommand{\id}{{\sf id}}
\newcommand{\id}{{1}}
\newcommand{\pp}{{\sf p}}
\newcommand{\mm}{{\sf m}}
\newcommand{\qq}{{\sf q}}
\newcommand{\COMP}[3]{{\sf comp}~#1~#2~#3}
\newcommand{\comp}{{\sf comp}}
\newcommand{\hcomp}{{\sf hcomp}}
\newcommand{\genComp}{{\sf Comp}}
\newcommand{\pres}{{\sf pres}}
\newcommand{\extend}{{\sf extend}}
\newcommand{\eq}{{\sf equiv}}

\newcommand{\Transp}{{\sf fill}}
\newcommand{\Glue}{{\sf Glue}}
\newcommand{\glue}{{\sf glue}}
\newcommand{\Comp}{{\sf fill}}
% Marc's macros
\newcommand{\op}[1]{#1^\mathit{op}}
\newcommand{\set}[1]{\{#1\}} 
\newcommand{\es}{\emptyset}
\newcommand{\lto}{\longmapsto}
\newcommand{\rup}[1]{#1{\uparrow}}
\newcommand{\rdo}[1]{#1{\downarrow}}
\newcommand{\rupx}[1]{#1{\uparrow_{x}}}
\newcommand{\rdox}[1]{#1{\downarrow_{x}}}
\newcommand{\rupxy}[1]{#1{\uparrow_{x,y}}}
\newcommand{\rdoxy}[1]{#1{\downarrow_{x,y}}}
\newcommand{\rupyx}[1]{#1{\uparrow_{y,x}}}
\newcommand{\rdoyx}[1]{#1{\downarrow_{y,x}}}
\newcommand{\del}[1]{}
\newcommand{\ul}[1]{\underline{#1}}
\newcommand{\bind}[2]{{\langle}#1{\rangle}#2}
\newcommand{\lam}[2]{{\langle}#1{\rangle}#2}
\newcommand{\make}[1]{{\langle}#1{\rangle}}
\newcommand{\OO}{O}
\newcommand{\many}[2]{{#1_1},\ldots,{#1_#2}}

% end Marc's macros

\begin{document}

\title{Proof theory of coherent theories}

\author{}
\date{}
\maketitle

\begin{abstract}
We give a constructive proof of Skolem's Theorem for coherent logic
and a new negative answer to a question by Wraith.
\end{abstract}

%\rightfooter{}

\section*{Introduction}
Skolemization is the replacement of an axiom of the form
\begin{equation}\label{allexist}
\forall\many{x}{n}~\exists y~\phi(\many{x}{n},y)
\end{equation}
by one of the form
\begin{equation}\label{allskol}
\forall\many{x}{n}~\phi(\many{x}{n},f_\phi(\many{x}{n})),
\end{equation}
where $f_\phi$ is a fresh function symbol, also called a Skolem function.

Clearly, (\ref{allskol}) is stronger than (\ref{allexist}).
Nevertheless, Skolem's Theorem states that skolemization is \emph{conservative}
for classical first-order logic. This means that classical consequences of
(\ref{allskol}) not containing $f_\phi$ already follow from (\ref{allexist}).

Semantically, Skolem's Theorem follows from the observation that
any model of (\ref{allexist}) can be extended with an interpretation
of $f_\phi$ to satisfy (\ref{allskol}). This simple, elegant argument has two
drawbacks. First, the interpretation of $f_\phi$ uses the Axiom of Choice.
Second, the argument does not explain how to transform proofs using
(\ref{allskol}) into proofs using (\ref{allexist}).
Of course, the mere existence of such a proof transformation
follows from the semantical argument by applying soundness and completeness.

From the proof theoretic point of view one would like to understand
in a combinatorial way how to transform proofs using (\ref{allskol}) into proofs
using  (\ref{allexist}).  Here and below we restrict attention to
proofs with conclusion not containing $f_\phi$. After all, such proofs are finite
combinatorial objects and will not use the skolem function in its entirety.
Such a proof transformation exists and has been defined by Maehara~\cite{Maehara}. 
A modern account can be found in, for example, Gallier~\cite[Ch.~7]{Gallier}.
These proof transformations are quite complicated.
First, the transformed proofs can be much longer than the original, see \cite{xxx}.
Second, as we shall see below, in certain cases the transformed proofs must use
the law of the excluded middle even if the original proof does not.

One natural question is whether Skolem's Theorem holds for other logics as well.
We first consider constructive logic. Surprisingly, Skolem's Theorem fails
for constructive logic, a result that is due to Minc?~\cite{xxx}.
For simplicity, take $n=1$ in (\ref{allexist}) and (\ref{allskol}),
and let $\phi$ be atomic. Consider the sentence
\begin{equation}\label{all2exist2}
\forall x_1,x_2~\exists y_1,y_2.~\phi(x_1,y_1) \land \phi(x_2,y_2) \land (x_1 = x_2 \to y_1 = y_2)
\end{equation}
Clearly,  (\ref{all2exist2}) follows from (\ref{allskol}) by taking $y_i = f_\phi(x_i)$.
In an attempt to prove (\ref{all2exist2}) from (\ref{allexist}), let $x_1,x_2$ be given.
Using (\ref{allexist}) we can get $y_1,y_2$ satisfying the first two conjuncts
in (\ref{all2exist2}). However, (\ref{allexist}) does not guarantee the third conjunct.
For this we would have to decide $ x_1 = x_2 \lor x_1 \neq x_2$ before the
application of (\ref{allexist}). In constructive logic, equality is in general not decidable.
We shall prove below that indeed (\ref{all2exist2}) does not follow from  (\ref{allexist})
in constructive logic.
Therefore Skolem's Theorem fails there.


\section{Coherent theories}
 
We use letters $r,s,t,\dots$ for terms and $x,y,z,u,\dots$ for variables.

\medskip

 We consider a theory $T$ with axioms of the form
$$
\Delta_0(\vec{y})~\vdash_{\vec{y}}~
\exists \vec{z_1}\Delta_1(\vec{y},\vec{z_1})\vee\dots\vee\exists \vec{z_n}\Delta_n(\vec{y},\vec{z_n})
$$
where $\Gamma(\vec{y}), \Delta(\vec{y}),\dots$ denote conjunctions of atomic formulae
all free variables of which are in $\vec{y}$, thought of as a finite sets of atomic formulae.
Such sets, as well as the sequences of variables, can be empty, 
and we can have axioms with $n=0$. 
We write $\Gamma,\Delta$ for the union of $\Gamma$ and $\Delta$.

 We define inductively when we have $\Gamma\vdash_{\vec{x}}^T A$ where $A$ is a atomic formula
and all free variables of $\Gamma,A$ are in $\vec{x}$. There are two clauses:

\begin{enumerate}
\item (base case) $A$ is in $\Gamma$ 

\item (step case) there is an axiom 
$$
\Delta_0(\vec{y})~\vdash_{\vec{y}}~
\exists \vec{z_1}\Delta_1(\vec{y},\vec{z_1})\vee\dots\vee\exists \vec{z_n}\Delta_n(\vec{y},\vec{z_n})
$$
in $T$ and a sequence of terms $\vec{t}$ with variables in $\vec{x}$,
such that $\Delta_0(\vec{t})$ is a subset of $\Gamma$ and
we have 
$$
\Gamma,\Delta_i(\vec{t},\vec{z_i})\vdash^T_{\vec{x},\vec{z_i}}
\text{for all $i = 1,\dots,n$.}
$$
\end{enumerate}

In the special case $n=0$ this means that $\Gamma\vdash_{\vec{x}} A$ for any $A$ if $\Gamma$
includes a finite set $\Delta_0(\vec{t})$. This shows that $\bot$ (absurdity) in coherent logic
is represented by the empty disjunction, satisfying the Ex Falso rule. Dually, we have $\top$
in coherent logic represented by  the empty conjunction (set of atoms), 
which is true in any $\Gamma$.
% left out remark on inconsistency that I didn't understand

\section{Some Lemmas}

%% \begin{lemma}\label{rename}
%% If $\sigma$ is a renaming on variables and
%% $\Gamma\vdash_{\vec{x}}^T A$ then $\Gamma\sigma\vdash_{\vec{x},z}^T A\sigma$.
%% \end{lemma}

\begin{lemma}\label{weak}
If $\Gamma\vdash_{\vec{x}}^T A$ and all free variables of $\Sigma$ are in $\vec{x}$ and
$\Gamma\subseteq \Sigma$ then $\Sigma\vdash_{\vec{x}}^T A$.
\end{lemma}

\begin{lemma}\label{subst1}
If $\vec{t} = t_1,\dots,t_n$ are terms using free variables in $\vec{y}$ and
$\Gamma\vdash_{x_1,\dots,x_n}^T A$ then $\Gamma(\vec{t})\vdash_{\vec{y}}^T A(\vec{t})$.
\end{lemma}

 Let $r$ be a term, and $u$ a variable, we denote by $\rho = \rho_{r,u}$ the substitution of $r$ by $u$
in an atomic formula. It is defined by taking $P(\vec{t})\rho = P(\vec{t}\rho)$ and
on terms
\begin{itemize}
\item $s\rho = u$ if $s = r$ otherwise
\item $f(\vec{t})\rho = f(\vec{t}\rho)$
\item $x\rho = x$
\end{itemize}

\begin{lemma}\label{subst2}
If all free variables of $r$ are in $\vec{x}$ and
$\Gamma\vdash_{\vec{x}}^T A$ then $\Gamma\rho_{r,u}\vdash_{\vec{x},u}^T A\rho_{r,u}$.
\end{lemma}

\begin{proof}
By induction on the proof of $\Gamma\vdash_{\vec{x}}^T A$.

If $A$ is in $\Gamma$ then $A\rho$ is in $\Gamma\rho$.

If there is a rule
$$
\Delta_0(\vec{y})~\vdash_{\vec{y}}~
\exists \vec{z_1}\Delta_1(\vec{y},\vec{z_1})\vee\dots\vee\exists \vec{z_n}\Delta_n(\vec{y},\vec{z_n})
$$
in $T$ such that $\Delta_0(\vec{t})$ is a subset of $\Gamma$ and
we have $\Gamma,\Delta_i(\vec{t},\vec{z_i})\vdash^T_{\vec{x},\vec{z_i}} A$ for all $i = 1,\dots,n$
where $\vec{t}$ are terms using variables in $\vec{x}$ then we have by induction
$\Gamma\rho,\Delta_i(\vec{t},\vec{z_i})\rho\vdash^T_{\vec{x},\vec{z_i},u} A\rho$ for all $i = 1,\dots,n$
Since $z\rho = z$ for all variable $z$ in $\vec{z_i}$, we get
$\Gamma\rho,\Delta_i(\vec{t}\rho,\vec{z_i})\vdash^T_{\vec{x},\vec{z_i},u} A\rho$ for all $i = 1,\dots,n$
and hence $\Gamma\rho\vdash^T_{\vec{x},u} A\rho$ as desired.
\end{proof}

\begin{lemma}\label{cut}
If $\Gamma\vdash_{\vec{x}}^T A$ and $\Delta,A\vdash_{\vec{x}}^T B$ then
$\Gamma,\Delta\vdash_{\vec{x}}^T B$
\end{lemma}

\begin{proof}
By induction on the proof of $\Gamma\vdash_{\vec{x}}^T A$. If $A$ belongs to $\Gamma$ we can
conclude by Lemma \ref{weak}. 
Otherwise we have a rule
$$
\Delta_0(\vec{y})~\vdash_{\vec{y}}~
\exists \vec{z_1}\Delta_1(\vec{y},\vec{z_1})\vee\dots\vee\exists \vec{z_n}\Delta_n(\vec{y},\vec{z_n})
$$
in $T$ such that $\Delta_0(\vec{t})$ is a subset of $\Gamma$ and
$\Gamma,\Delta_i(\vec{t},\vec{y_i})\vdash_{\vec{x},\vec{y_i}} A$ for all $i$.
By induction, we have $\Gamma,\Delta,\Delta_i(\vec{t},\vec{y_i})\vdash_{\vec{x},\vec{y_i}} B$ for all $i$
and hence $\Gamma,\Delta\vdash_{\vec{x}}^T B$ as desired.
\end{proof}

\section{Proving disjunctive and existential statements}

 We generalize the definition of $\Gamma\vdash_{\vec{x}}^T A$ to 
$\Gamma\vdash_{\vec{x}}^T \Delta$ where $\Delta$ is a finite set of atomic formulae, the free
variable of which are among $\vec{x}$, which is now thought of as a disjunction. The clauses are

\begin{enumerate}
\item $\Delta$ meets $\Gamma$

\item there is a rule
$$
\Delta_0(\vec{y})~\vdash_{\vec{y}}~
\exists \vec{z_1}\Delta_1(\vec{y},\vec{z_1})\vee\dots\vee\exists \vec{z_n}\Delta_n(\vec{y},\vec{z_n})
$$
in $T$ such that $\Delta_0(\vec{t})$ is a subset of $\Gamma$ and
we have $\Gamma,\Delta_i(\vec{t},\vec{z_i})\vdash^T_{\vec{x},\vec{z_i}} \Delta$ for all $i = 1,\dots,n$
where $\vec{t}$ are terms using variables in $\vec{x}$.
\end{enumerate}

 We define $\vdash_{y_0,y_1}^T \exists u~P(u,y_0,y_1)$ to mean that there are a finite number
of terms $r_1(y_0,y_1),\dots,r_n(y_0,y_1)$ such that
$\vdash_{y_0,y_1}^T P(r_1,y_0,y_1),\dots,P(r_n,y_0,y_1)$.

%% \begin{lemma}
%% If $\vdash_{y_0,y_1}^T \exists u~P(u,y_0,y_1)$ and all free variables of $\Gamma,A,t_0,t_1$ are in
%% $\vec{x}$
%% and $\Gamma,P(z,t_0,t_1)\vdash_{\vec{x},z}^T A$ then $\Gamma\vdash_{\vec{x}}^T A$.
%% \end{lemma}

\section{Addition of existential rules}

 We assume that there are a finite number
of terms $r_i(y_0,y_1)$ such that
$\vdash_{y_0,y_1}^T P(r_1,y_0,y_1),\dots,P(r_n,x,y)$ (i.e. $\vdash_{y_0,y_1}^T \exists u~P(u,y_0,y_1)$).
We let $T'$ be the theory where we add the rule
$$
\vdash_{y_0,y_1}~\exists u~P(u,y_0,y_1)
$$

\begin{lemma}
If $\Gamma\vdash_{\vec{x}}^{T'} A$ then $\Gamma\vdash_{\vec{x}}^T A$.
\end{lemma}


\begin{proof}
By induction on the proof of $\Gamma\vdash_{\vec{x}}^{T'} A$. The only complex
case is if we use the new rule. In this case we have terms $t_0,t_1$ using only
free variables in $\vec{x}$ and $\Gamma,P(u,t_0,t_1)\vdash_{\vec{x},u}^T A$ by induction.
Using Lemma \ref{subst1} we get
$\Gamma,P(r_i(t_0,t_1),t_0,t_1)\vdash_{\vec{x}} A$ for $i = 1,\dots,n$.
We also have $\vdash_{\vec{x}}^T P(r_i(t_0,t_1),t_0,t_1)$, and we can conclude by Lemma \ref{cut}.
\end{proof}


\section{Elimination of function symbols}

 We assume that we have the rule $\vdash_{y_0,y_1}\exists u~P(u,y_0,y_1)$ in $T$.
(Using the previous section, we can assume instead only that this rule is {\em derivable}
in $T$.)
We introduce $f$ a binary function
symbol which is {\em not} in the language of $T$. We let $T_0$ be the theory like $T$ but where we
add the symbol function $f$ and $T'$ the extension of $T_0$ with the rule
$$
\vdash_{y_0,y_1}~P(f(y_0,y_1),y_0,y_1)\leqno{(*)}
$$

 The goal of this section is to prove that $T'$ is a conservative extension of $T_0$. (It is clear
that $T_0$ is a conservative extension of $T$ if we have a constant symbol in $T$ since we can always
interpret the function by a constant function.) The main idea is that if we use $f$ and the rule
$(*)$ with $P(f(s,t),s,t)$ we should be able to replace $f(s,t)$ by a fresh variable 
$P(u,s,t)$ and then we can use the fact that $\exists u~P(u,s,t)$ is provable.
This idea is cleverly expressed in the crucial Lemma \ref{crucial} (which comes from \cite{Maehara}).

% We define a $f$-term of $\Gamma$ to be any term of the form $f(t_0,t_1)$ which appears
%in $\Gamma$.

\begin{lemma}\label{simpl}
If $\Gamma,P(f(s,t),s,t)\vdash_{\vec{x}}^{T_0} A$ and
$f(s,t)$ is not in $\Gamma,A$ then $\Gamma\vdash_{\vec{x}}^{T_0} A$.
\end{lemma}

\begin{proof}
We take $r = f(s,t)$ and we use Lemma \ref{subst2}. We get
$\Gamma,P(u,s,t)\vdash_{\vec{x},u}^{T_0} A$. Since we also have
$\vdash_{\vec{x}}\exists u~P(u,x,t)$ we get  $\Gamma\vdash_{\vec{x}}^{T_0} A$.
\end{proof}

\begin{corollary}\label{simpl1}
If $\Gamma,P(f(s_1,t_1),s_1,t_1),\dots,P(f(s_n,t_n),s_n,t_n)\vdash_{\vec{x}}^{T_0} A$ and
$f(s_1,t_1),\dots,f(s_n,t_n)$ are not in $\Gamma,A$ then $\Gamma\vdash_{\vec{x}}^{T_0} A$.
\end{corollary}


\begin{lemma} \label{crucial} (crucial Lemma) 
If $\Gamma\vdash_{\vec{x}}^{T'} A$ then $\Gamma,\Sigma\vdash_{\vec{x}}^{T_0} A$ where
$\Sigma$ is a set of formulae $P(f(s,t),s,t)$ where $f(s,t)$ is a term which
occurs in $\Gamma,A$.
\end{lemma}

\begin{proof}
By induction on the proof of $\Gamma\vdash_{\vec{x}}^{T'} A$.

\medskip

 The first case if if $A$ is in $\Gamma$ then $A$ is in $\Gamma,\Sigma$ and so 
$\Gamma,\Sigma\vdash_{\vec{x}}^{T_0} A$.

\medskip

 The second case is if there is a rule
$$
\Delta_0(\vec{y})~\vdash_{\vec{y}}~
\exists \vec{z_1}\Delta_1(\vec{y},\vec{z_1})\vee\dots\vee\exists \vec{z_n}\Delta_n(\vec{y},\vec{z_n})
$$
in $T$ with $\vec{t}$ terms using free variables in $\vec{x}$ and
we have $\Delta_0(\vec{t})$ subset of $\Gamma$ and we have
$\Gamma,\Delta_i(\vec{t},\vec{z_i})\vdash_{\vec{x},\vec{z_i}}^{T'} A$ for $i=1,\dots,n$.
By induction we have $\Gamma,\Delta_i(\vec{t},\vec{z_i}), \Sigma_i\vdash_{\vec{x},\vec{z_i}}^{T_0} A$
for $i = 1,\dots,n$, where $\Sigma_i$ is a set of formulae $P(f(t_0,t_1),t_0,t_1)$ where
$f(t_0,t_1)$ occurs in $\Gamma,A,\Delta_i(\vec{t},\vec{z_i})$.
Since $\Delta(\vec{y},\vec{z_i})$ does not mention $f$ and all free variables of $\vec{t}$ are
in $\vec{x}$ it follows that all free variables of $\Sigma_i$ are in $\vec{x}$.
It follows that we have
$\Gamma,\Sigma_1,\dots,\Sigma_n\vdash_{\vec{x}}^{T_0} A$.
Then using Corollary \ref{simpl1}, we get the required conclusion.

\medskip

 The last case is that we have used the rule $(*)$ and
we have $\Gamma,P(f(t_0,t_1),t_0,t_1)\vdash_{\vec{x}}^{T'} A$. By induction we get
$\Gamma,P(f(t_0,t_1),t_0,t_1),\Sigma \vdash_{\vec{x}}^{T_0} A$ where
$\Sigma$ is a set of all formulae $P(f(s,t),s,t)$ where $f(s,t)$ occurs
in $\Gamma,A,P(f(t_0,t_1),t_0,t_1)$. We then conclude using Corollary \ref{simpl1} as in the
previous case.
\end{proof}

 For the next result, we assume the theory to have at least one constant symbol.

\begin{theorem}
If $\Gamma,A$ does not mention $f$ and $\Gamma\vdash_{\vec{x}}^{T'} A$ then
$\Gamma\vdash_{\vec{x}}^T A$.
\end{theorem}

\begin{proof}
By the previous crucial Lemma we get $\Gamma\vdash_{\vec{x}}^{T_0} A$. We then conclude
$\Gamma\vdash_{\vec{x}}^T A$ by replacing all terms using $f$ by the given constant.
\end{proof}

%%  If we have an equality symbol, we consider that we have added all equality axioms for
%% predicate and function symbols in the theory. It follows from this Theorem that we
%% can eliminate function symbols whenever we have unique existence. 

\section{Addition of equality}

 The same reasoning seems also to work with the addition of the rule
$$x=x',~ y = y'\rightarrow f(x,y) = f(x',y')$$
for the new function symbol $f$.

 We only have to consider the new case where we deduce $\Gamma\vdash_{\vec{x}}^{T'} A$ from 
$\Gamma,f(s_0,t_0) = f(s_1,t_1)\vdash_{\vec{x}}^{T'} A$ and $s_0 = s_1$ and
$t_0 = t_1$ in $\Gamma$. We then have by induction
$\Gamma,f(s_0,t_0) = f(s_1,t_1),\Sigma \vdash_{\vec{x}}^{T_0} A$
where  $\Sigma$ is a set of formulae $P(f(s,t),s,t)$ where $f(s,t)$ occurs
in $\Gamma,f(s_0,t_0) = f(s_1,t_1),A$ and we conclude using Corollary \ref{simpl1}.

\medskip

 It is remarkable that this proof transformation is quite simple (no combinatorial explosion), while
the introduction of an extensional Skolem function is intuitively problematic. For instance,
we do have for real numbers $\vdash_x \exists n~n\leqslant x$ but there is no Skolem function
giving $n$ from $x$. While this remark is only heuristic, we provide next two examples where there
is {\em no} Skolem functions in some natural models of the theory $T$.

\section{Examples}

\subsection{Algebraically closed field}

 We can take $T$ be the theory of algebraically closed field. This theory
has for axioms the equational theory of rings together with the rules
$$
\vdash_x x = 0 \vee \exists y~(xy = 1)~~~~~~~~~~~~
\vdash_{x_1,\dots,x_n} \exists x~(x^n = x_1 x^{n-1} + \dots + x_n)
$$

 We can prove in this theory
$$
\vdash_{x} \exists u~ (x(1-ux) = 0)
$$
in this theory. Hence we can introduce a function $inv(x)$ with the axiom
$x(1 -inv(x) x) = 0$ and we get a conservative extension. 

 Another example with the same theory is to introduce a function $sqrt(x)$ with the axiom
$sqrt(x)^2 = x$. Since we have $\vdash_{x} \exists u~(u^2 = x)$, we also get a conservative
extension.

 For this example, notice that we have models of the theory (for instance the model described
in \cite{Mannaa}) where there is no square root function.

\subsection{Local rings}

 Another example is the the theory of local rings, given by the equational theory of rings
with the rules
$$
0=1\vdash~~~~~~~~~~~~~~~~~~~~~~~~~~~~\vdash_x inv(x)\vee inv(1-x)
$$
where $inv(x)$ means $\exists y~(xy = 1)$. 

 The generic model is given by the site of finitely presented rings with covering
$R\rightarrow R[1/x_i]$ for $1 = \ideal{x_1,\dots,x_n}$ (we can have $n = 0$ if $1=0$ in $R$).

 We can prove in this theory
$$
\vdash_x \exists y~z~(x y = 1~\vee ~(1-x) z = 1)
$$
hence it is possible in a conservative way
to introduce two skolem functions $J(x)$ and $K(x)$ with the rule
$$
\vdash_x xJ(x) = 1~\vee~(1-x)K(x) = 1
$$

 There are no such functions for the generic model since these functions should be defined
at all levels and $\ints$ is not a local ring.
(All functions on the generic local rings are polynomials but we don't need to use this fact.)

\medskip

 Despite the fact that the conservativity can be proved in a simple way, in the last case, we
have no Skolem function in the generic model, and thus, we cannot simply rely on a semantical
argument.

\section{Application to a question of Gavin Wraith}

 We take the language with only a binary relation $R$. The theory $T$ has the rule
$\vdash_x \exists y~R(x,y)$. As an application of what we have seen, we can add
in a conservative way a function symbol $f$ with the rule $\vdash _x R(x,f(x))$ and
get a theory $T'$ which is conservative over $T$ for {\em coherent} formulae. 

 This theory however is {\em not} conservative for arbitrary formulae since we can prove in
$T'$ the non coherent formula
$$
\exists y_1~y_2~(R(x_1,y_1)\wedge R(x_2,y_2)\wedge (x_1=x_2\rightarrow y_1=y_2))\leqno{(**)}
$$
which is not provable in $T$.

 We show that this theory is {\em not} valid in the generic model of $T$, answering
in this way a question of Gavin Wraith \cite{Wraith}. The generic model can be described
in the following way. It is a site model. The object of the site are relations (graphs)
$I,R_I$ on a finite set
and morphisms maps $f:I\rightarrow J$ which preserve the relation: if $R_I(i_0,i_1)$
then $R_J(f(i_0),f(i_1))$. A basic covering of $I,R_I$ is obtained by choosing an element $i$
in $I$ and adding a fresh element $I\rightarrow I,j$ and the relation $R(i,j)$.
The covariant functor $(I,R_I)\longmapsto I$ satisfies the sheaf condition and is the domain
of the generic model. We define $(I,R_I)\Vdash\varphi$ by induction on $\varphi$ as usual.
It can then be checked that the formula $(**)$ above is not forced.

\begin{thebibliography}{9}

\bibitem{DW}
G. Dowek and B. Werner.
\newblock{A constructive proof of Skolem theorem for constructive logic.}
\newblock{Manuscript, 2004.}

\bibitem{Gallier} 
J. Gallier, 
\newblock\emph{Logic for Computer Science: Foundations of Automatic Theorem Proving},
Wiley, 1986. Freely available online at: \url{www.cis.upenn.edu/~jean/gbooks/logic.html}

\bibitem{Maehara}
S. Maehara.
\newblock{The predicate calculus with $\epsilon$-symbol.}
Journal of the Mathematical Society of Japan, \textbf{7}(4):323--344, 1955.

\bibitem{Mannaa}
Th. Coquand and B. Mannaa.
\newblock{A Sheaf Model of the Algebraic Closure.}
\newblock{Proceedings of EPTCS, 2014.}

\bibitem{Skolem}
Th.~Skolem,
\newblock\emph{Logisch-kombinatorische Untersuchungen \"{u}ber
die Erf\"{u}llbarkeit % manual break
und Beweisbarkeit mathematischen S\"{a}tze
nebst einem Theoreme \"{u}ber dichte Mengen},
{Skrifter} I \textbf{4}:1--36, Det Norske Videnskaps-Akademi, 1920.
\newblock Also in: Jens Erik Fenstad, editor,
\emph{Selected Works in Logic by Th. Skolem}, pp.~103--136,
Universitetsforlaget, Oslo, 1970.

\bibitem{Wraith}
G. C. Wraith
\newblock{Intuitionistic algebra: some recent developments in topos theory.}
\newblock{Proceedings of the International Congress of Mathematicians (Helsinki, 1978), pp. 331–337.}

\end{thebibliography}

\end{document}      
                                                                                  
 
