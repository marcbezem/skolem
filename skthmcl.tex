
%NB one \newpage on title page!
\documentclass[12pt]{article}
\newcommand{\weg}[1]{\medskip}
%\pagestyle{empty}
\usepackage{url,a4,proof}
\usepackage{latexsym}
\usepackage{amstext}
\usepackage{amssymb}
%\usepackage[all]{xy}
\usepackage{theorem}

\newcommand{\ra}{\rightarrow}
\newcommand{\Ra}{\Rightarrow}
\newcommand{\lra}{\leftrightarrow}
\newcommand{\rsa}{\rightsquigarrow}
\newcommand{\goal}{\mathit{goal}}
\newcommand{\halt}{\mathit{halt}}
\newcommand{\tru}{\mathit{true}}
\newcommand{\fal}{\mathit{false}}
\newcommand{\dom}{\mathit{dom}}

\newcommand{\tp}{:}             % typing relation 
\newcommand{\ltp}{\!:\!}        % typing relation for lambda/pi abstractions
\newcommand{\ap}{.\,}           % application of lambda abstraction

\newcommand{\obj}[1]{\dot{#1}}
\newcommand{\E}{\mathit{E}}
\newcommand{\Nat}{\mathit{Nat}}
\newcommand{\bfc}{\pitchfork}

%\renewcommand{\vec}[1]{\overrightarrow{#1}}

%%%%%%%%%%%%%%%%%%%%%%%%%%%%%%%%%%%%%%%%%%%%%%%%%%%%%%%%%%%%%%%%%%%%%%%%%%%%%%%%

\newcommand{\many}[2]{{#1_1},\ldots,{#1_#2}}
\newcommand{\ls}{\lbrack\!\lbrack}
\newcommand{\rs}{\rbrack\!\rbrack}
\newcommand{\sem}[1]{\ls {#1} \rs}
\newcommand{\set}[1]{\{#1\}}
%\newcommand{\Noline}{\ar@{}}
%\UseComputerModernTips
\newcommand{\cl}[1]{\overline{#1}}

\newcommand{\Tf}{T_{\!f}}
\newcommand{\psif}{\psi_{\!f}}



%%%%%%%%%%%%%%%%%%%%%%%%%%%%%%%%%%%%%%%%%%%%%%%%%%%%%%%%%%%%%%%%%%%%%%%%%%%%%%%%

\begin{document}

\title{On Skolem's Theorem for Coherent Logic}
%
\author{Marc Bezem}
%\institute{Department of Computer Science, University of Bergen\\
%P.O.~Box~7800, N-5020 Bergen, Norway\\
%\email{bezem@ii.uib.no}}

\date{\it Dedicated to Roel de Vrijer on the occasion of his 60-th birthday}

\maketitle

\bibliographystyle{plain}


\section{Introduction}\label{sec:intro}
Assume one replaces, in some first-order theory $T$, an axiom of the form
\[\forall\many{x}{n}~\exists y~\phi(\many{x}{n},y)\]
by 
\[\forall\many{x}{n}~\phi(\many{x}{n},f_\phi(\many{x}{n})),\]
where $f_\phi$ is a so-called Skolem-function for $\phi$ not occurring in $T$.
This replacement, for example, for the purpose of eliminating existential quantifications,
seems a reckless strengthening, as the latter axiom simply will be false
for many interpretations of $f_\phi$. Reassuringly, 
Skolem's Theorem states that such a replacement 
is conservative in the sense that it doesn't allow one to prove more sentences
not involving $f_\phi$ than one already could prove in $T$.

Skolem's Theorem has an easy \emph{semantic} justification which goes as follows.
With $T$ as above and abbreviating 
$\psi:=\forall\vec{x}~\exists y~\phi(\vec{x},y)$,
let $\sigma$ be a sentence in which the Skolem-function $f_\phi$ doesn't occur. 
Let $\Tf$ be $T$ with $\psi$ replaced by
$\psif:=\forall\vec{x}~\phi(\vec{x},f_\phi(\vec{x}))$.
Assume there is a model of $T$ not satisfying $\sigma$. 
Let $D$ be the domain of this model and interpret
$f_\phi$ as a function picking for every $\many{x}{n}\in D$ a $y\in D$ such that
the interpretation of $\phi(\many{x}{n} ,y)$ is true. Clearly, this interpretation
of $f_\phi$ makes $\psif$ true. Moreover, interpreting $f_\phi$
doesn't change the interpretation of neither $T$ nor $\sigma$, as $f_\phi$
only occurs in $\psif$. It follows that adding the interpretation of
$f_\phi$ to the model of $T$ results in a model of $\Tf$
in which $\sigma$ is not satisfied.
Skolem's Theorem can now be obtained by contraposition: $\sigma$ is a logical
consequence of $T$ if $\sigma$ is so of $\Tf$.

The above semantic justification of Skolem's Theorem has two important
drawbacks. First, the construction of the interpretation of
$f_\phi$ actually requires the Axiom of Choice. Second, the semantic justification
doesn't in any way explain how to transform proofs in $\Tf$ to proofs in $T$.
For this reason proof-theorists have for a log time been interested in a \emph{syntactic}
justification of Skolem's Theorem \cite{Maehara}. By this we mean a proof transformation
which transforms any proof in $\Tf$ whose conclusion doesn't contain
a Skolem-function into a proof in $T$ with the same conclusion. 
A glance at \cite[Ch.~7]{Gallier} makes it
immediately clear why most textbooks stick to the semantic approach:
the syntactic proof transformations appear to be rather complicated, 
to an extent that most readers will prefer the semantic justification.
Nevertheless, syntactic approaches to Skolem's Theorem
have several advantages, for example, a direct proof explaining why $\sigma$
is a logical consequence of $T$. Moreover, such a proof can be verified directly, 
without the formalization of the semantic justification of Skolem's Theorem
(which would involve the Axiom of Choice).
%For the latter purpose it is important that the size of the proofs does not go out of hand.

Now, as the technical difficulties of the syntactic approach to Skolem's Theorem
seem to depend on the inference system chosen, it may be easier to depart
from a simpler, yet fully-expressive logic.
This is exactly the reason why we here propose to study Skolem's Theorem
in Coherent Logic. 


\section{Coherent Logic}\label{sec:cl}

Coherent logic (CL) is
the fragment of first-order logic consisting of implicitly universally quantified
implications of the following form:
\begin{equation}%\label{CLformat}
  {A_1 \wedge \cdots \wedge A_n} 
  \ra 
  {\exists \vec{x}_1.\,\mathbf{C}_1 \vee \cdots \vee \exists \vec{x}_m.\,\mathbf{C}_m}
\end{equation}
where the $A_i$ are first-order atoms and the $\mathbf{C}_{j}$ are 
conjunctions of such atoms. We use some obvious
notational optimizations to improve readability:
if $n=0$, then we leave out $\ra$ altogether;
if $m=0$, then we write $\bot$ (\textit{falsum}) 
to denote the empty disjunction; 
empty existential quantifications are left out.
A coherent theory is a finite set of formulas of the form~(1). %\eqref{CLformat}.
Closed atoms, i.e., atoms without free variables, are called \emph{facts}.

Let $T$ be a coherent theory.
The set $\Delta^T(X\vdash F)$ of derivations in $T$ of a fact $F$
from a set of facts $X$ is inductively defined by the following two rules (explained below): 
\begin{eqnarray*}
  \mbox{\infer[F\in X]{~F~}{~X\vphantom{\delta_1}~}}
  \quad\quad
  \mbox{
    \infer[\mathbf{A}\subseteq X]{F}{
      X & \mathbf{A}\ra \mathbf{D}\quad\delta_1 & \cdots\quad\delta_m%\nonumber\\
    }
  }
\end{eqnarray*}
In words, the base case (left) applies when the goal $F$ is in the set of facts $X$.
The step case applies when $\mathbf{A}\ra \mathbf{D}$
is a closed\footnote{Both the vocabulary of $T$ and of $X$ can be used.}
instance of a formula in $T$ whose antecedent
is satisfied by $X$, expressed by $\mathbf{A}\subseteq X$. 
The subderivations $\delta_i$ correspond to the disjuncts in $\mathbf{D}$
in the following way. If $\mathbf{D}=
{\exists \vec{x}_1.\,\mathbf{C}_{1} \vee \cdots \vee \exists \vec{x}_m.\,\mathbf{C}_{m}}$,
every subderivation $\delta_i$ should be in $\Delta^T(X,\cl{\mathbf{C}_{i}}\vdash F)
~(1\leq i\leq m)$. Here $X,\cl{\mathbf{C}_{i}}$ is the set of facts $X$
extended with the atoms in $\mathbf{C}_{i}$, with the variables $\vec{x}_i$
replaced by fresh constants.
If $\mathbf{D}$ is $\bot$ there are no subderivations.  

As an example we give the derivation of $r$ in the coherent theory with axioms

\[ p\vee\exists y~q(y) \quad\quad p\ra\bot \quad\quad q(x)\ra r\] 
(we don't denote side conditions of rule applications):
\[
\infer{r\vphantom{|}}
{\emptyset & p\vee \exists y~q(y) & \infer{r\vphantom{|}}{\set{p} & p\ra\bot} & \infer{r\vphantom{|}}
 {\set{q(c)} & q(c)\ra r & \infer{r\vphantom{|}}
  {\set{q(c),r}
  }
 }
}
\]
The rightmost inference is the only base case, all other
inferences are step cases. Note that the step using $p\ra\bot$
has no subderivations but is not a base case.

There are two views on CL's proof theory.
One is to view the above rules as a 
natural deduction-style system of inference.
The step case is actually a combination of
instantiation (of the formulas in $T$), Modus Ponens (combining $X$
with $\mathbf{A}\ra \mathbf{D}$), disjunction elimination 
(the subderivations for each disjunct in $\mathbf{D}$)
and existential elimination (the fresh constants in $\cl{\mathbf{C}_i}$).
The proof theory is complete.

The other view is that of forward reasoning.
This can also be observed in the example above: 
we start with the empty set of facts, we split in
two branches with sets $\set{p}$ and $\set{q(c)}$, respectively,
and finally extend the latter to $\set{q(c),r}$. 
For every set of facts $X$, every derivation in $\Delta^T(X\vdash F)$
ends in $F$, so denoting $F$ is rather redundant.
If we apply this to the example above, only denoting the sets of
facts (and shifting a bit), the following tree emerges:

\[
\infer{\emptyset}
{
\infer{\set{p}}{(\bot)} & \infer{\set{q(c)}}{\set{q(c),r}}
}
\]
The line above the set $\set{p}$ is expressing that this is 
not a base case, but an induction case with zero subderivations
(in which $r$ doesn't need to be checked). 
Alternatively, we could have written $\bot$ above this line
to close the branch.

Observe that the axioms are used more or less as rewrite roels
to generate new facts
from already known ones, distinguishing cases
for each disjunct in the consequent and introducing witnesses
in the case of existential quantifiers.
The view of forward reasoning will make it particularly easy to
devise a proof transformation eliminating Skolem-functions.
 
\weg{and reasoning in CL is constructive
in the sense of intuitionistic logic.
Completeness proofs can be found in~\cite{Beze:Coqu:04,Beze:Coqu:05}.
We have implemented the CL proof procedure in Prolog, 
see~\cite[\url{CL.pl}]{Beze:Hend:05}. 
The implementation generates {\coq} proof scripts.
This does not increase the size of the trusted core of {\coq}.
Elaborated examples of the reasoning mechanism are given in 
Section~\ref{sec:machine} and in the appendix.
}%\endweg

\section{Elimination of Skolem-functions}

As a fragment of first-order predicate logic, CL is very expressive. 
For example, the following two coherent formulas define
$np$ as the negation of the predicate $p$:
\[p(x,y)\vee np(x,y) \quad\quad p(x,y)\wedge np(x,y)\ra\bot\]
By introducing new predicates and defining them equivalent to (sub)formulas
one can easily prove that every first-order theory has a definitional extension
which is equivalent to a coherent theory.

Note that both $\forall\vec{x}~\exists y~ \phi(\vec{x},y)$
and $\forall\vec{x}~\phi(\vec{x},f_\phi(\vec{x}))$
already are coherent if $\phi$ is a predicate.
Moreover, by the last remark in the previous paragraph it suffices to
restrict attention to predicates only; for simplicity
we take the predicate to be binary.

Let $T$ be a coherent theory containing
$\psi:=\forall x~\exists y~p(x,y)$ and let $\Tf$ be $T$ with
$\psi$ replaced by $\psif:=\forall x~p(x,f(x))$,
where $f$ doesn't occur in $T$.
Assume some fact $q$ not containing $f$ can be inferred
from $\Tf$ by a proof $\delta$ 
in the inference system from Section~\ref{sec:cl}.

Let us take the view of forward ground reasoning on $\delta$.
If $f$ doesn't occur in $\delta$, then $\psif$ has
not been used and $\delta$ is actually a proof in $T$
(even without $\psi$). Otherwise, there is a term $f(t)$ which
occurs in $\delta$. 
%In other words, $f(t)$ is a minimal $f$-term in $\delta$.
There are two ways in which $f(t)$ can show up in $\delta$:
as a (sub)term substituted for a variable in an axiom of $T$, or in
$p(t,f(t))$ when applying $\psif$. 
One is tempted to take a fresh constant $t_p$ and
replace $f(t)$ by $t_p$ everywhere in $\delta$.
The problem is that one may not get a valid proof using $\psi$ in this way.
%$exists y~p(x,y)$
As an example of what can go wrong in the proof transformation,
consider the following coherent theory $\Tf$:

\[q(x) \quad\quad p(x,f(x)) \quad\quad p(x,y),q(y)\ra r(a)\] 
Since there are no disjunctions, proofs are linear
and can be written horizontally. Consider the following proof of $r(a)$:

\[\emptyset \mid \set{q(f(a))} \mid \set{q(f(a)),p(a,f(a))} \mid \set{q(f(a)),p(a,f(a)),r(a)} \] 
Replacing the term $f(a)$ by a new constant $a_p$ in the proof we get:

\[\emptyset \mid \set{q(a_p)} \mid \set{q(a_p),p(a,a_p)} \mid \set{q(a_p),p(a,a_p),r(a)} \] 
Unfortunately, this is \emph{not} a proof using $\exists y~p(x,y)$
instead of $p(x,f(x))$, since the constant $a_p$ in $p(a,a_p)$ is not fresh.
The solution is to let the proof start with the application of $p(x,f(x))$:

\[\emptyset \mid \set{p(a,f(a))} \mid \set{p(a,f(a)),q(f(a))} \mid \set{p(a,f(a)),q(f(a)),r(a)} \]
Replacing $f(a)$ by $a_p$ now yields a correct proof:

\[\emptyset \mid \set{p(a,a_p)} \mid \set{p(a,a_p),q(a_p)} \mid \set{p(a,a_p),q(a_p),r(a)} \]
of $r(a)$ in the coherent theory $T$:

\[q(x) \quad\quad \exists y~p(x,y) \quad\quad p(x,y),q(y)\ra r(a)\]

So the idea is to have the applications of $p(x,f(x))$ as early as possible in the proof.
The question is now: what does `as early as possible' mean? If the term $f(t)$
is built from the vocabulary of $\Tf$ the above method works correctly.
But what if $t$ contains a constant that is introduced as a fresh constant at some
place in the proof? 
As the examples get a little longer, we make use of the fact that the sets
of facts are increasing and only denote the newly inferred atoms. At any point in
the proof the set of facts can be reconstructed by collecting all facts occurring on the left.
Consider the following proof in the coherent theory $\Tf$:

\[q(f(f(a))) \mid p(a,f(a)) \mid p(f(a),f(f(a))) \mid r(a) \] 
By the above proof transformation we correctly eliminate $p(a,f(a))$:

\[p(a,a_p) \mid q(f(a_p)) \mid p(a_p,f(a_p)) \mid r(a) \] 
Note that $p(a_p,f(a_p))$ contains a constant which doesn't occur
in $T$, but has been introduced in the first step of the proof above.
It is therefore not possible to start the proof by inferring $p(a_p,f(a_p))$.
The phrase `as early as possible' means `as soon as all constants occurring
in the term have been introduced', in this case as the second atom.
Replacing $f(a_p)$ by a fresh constant then indeed leads to a proof in $T$:

\[p(a,a_p) \mid p(a_p,a_{pp})) \mid q(a_{pp}) \mid r(a) \] 

We now sketch the proof transformation in general.
Assume a proof in a coherent theory $\Tf$ contains a term $f(t)$.
If the proof doesn't contain a fact $p(t,f(t))$,
then all occurrences of $f(t)$ have been introduced
by substitution and we can replace them without difficulty
by $t$ (which is the only closed term of which we can
be certain that it exists!).
Since $f$ occurs in no other axiom than $p(x,f(x))$,
all inferences remain valid. 

If, on the other hand, the proof contains a fact $p(t,f(t))$,
then we move this application of $p(x,f(x))$ towards the root
of the tree to the first position where all constants
occurring in $t$ have been introduced. In the subtree above
this position (inclusive), we replace all occurrences of $f(t)$
by a fresh constant, and the application of $p(x,f(x))$ is
replaced by an application of $\exists y~p(x,y)$. Possible
multiple occurrences of $p(t,f(t))$, for example, in different
branches, should all be removed. After this operation, again all
inferences are valid.  

Proceeding in this way we get after finitely many steps
a proof in which $f$ doesn't occur, that is, a proof in $T$.
Moreover, all facts not containing $f$, including the conclusion,
are unaffected by the substitutions.
Thus we have proved Skolem's Theorem syntactically.
Note that the proof transformation is quite robust in the sense
that it doesn't matter whether or not $f(t)$ contains
other occurrences of $f$ and/or other functions.
Also, branching poses no extra difficulties over the linear case.

\section*{Acknowledgements}
The author wants to thank Gilles Dowek for several discussions
on Skolem's Theorem over the past decade.

\bibliographystyle{plain}
\begin{thebibliography}{99}

\bibitem{Gallier} 
J. Gallier, 
\newblock\emph{Logic for Computer Science: Foundations of Automatic Theorem Proving},
Wiley, 1986. Freely available online at: \url{www.cis.upenn.edu/~jean/gbooks/logic.html}

\bibitem{Maehara}
S.~Maehara, 
\newblock\emph{The predicate calculus with $\varepsilon$-symbol},
Journal of the Mathematical Society of Japan, \textbf{7}(4):323--344, 1955.

\end{thebibliography}

\end{document}


\bibitem{Skolem}
Th.~Skolem,
\newblock\emph{Logisch-kombinatorische Untersuchungen \"{u}ber
die Erf\"{u}llbarkeit % manual break
und Beweisbarkeit mathematischen S\"{a}tze
nebst einem Theoreme \"{u}ber dichte Mengen},
{Skrifter} I \textbf{4}:1--36, Det Norske Videnskaps-Akademi, 1920.
\newblock Also in: Jens Erik Fenstad, editor,
\emph{Selected Works in Logic by Th. Skolem}, pp.~103--136,
Universitetsforlaget, Oslo, 1970.


Every inference of $p(t,f(t))$ using $\psif$
can be replaced by an inference of
$p(t,t_p)$ using $\psi$. 
All other occurrences of $f(t)$ are introduced
by substituting a term $t'(f(t))$ for a variable, in which case we 
equally well could have substituted $t'(t_p)$ instead of $t'(f(t))$.
Since $f$ occurs in no other axiom than $\psif$,
all inferences remain valid. Furthermore, all facts
not containing $f$ are unaffected by the substitution,
and this holds in particular for the conclusion $q$.
Thus we have managed to eliminate at least one occurrence of $f$ from
the proof, and can continue this process until all occurrences
of $f$ have been eliminated. This will result in a proof of $q$
from $T$. Note that the resulting proof has exactly the same
tree-structure as the original one.



consisting of:
\[
p(x),~
q(a,a_p)\vee q(a,c),~
q(x,y)\ra q(y,y),~
p(x),q(x,x)\ra r
\]

The following is a (purposedly contrived) proof of $r$ from $T$ using 
the Skolem-axiom for $q$ in the 4-th line from below, left and right:
\[
\infer{\emptyset}
{
\infer{\set{p(f(a))}}
 {
%left
\infer{\set{p(f(a)),q(a,a_p)}}
  {
\infer{\set{p(f(a)),q(a,a_p),q(a,f(a))}}
   {
\infer{X_b \ cup \set{q(f(a),f(a))}}
    {
       X_b \ cup \set{q(f(a),f(a)),r}
    }
   }
  }
&
%right
\infer{\set{p(f(a)),q(a,c)}}
  {
\infer{\set{p(f(a)),q(a,c),q(a,f(a))}}
   {
\infer{X_c \ cup \set{q(f(a),f(a))}}
    {
       X_c \ cup \set{q(f(a),f(a)),r}
    }
   }
  }
 }
}
\]

Here $X_b$ abbreviates $p(f(a)),q(a,a_p),q(a,f(a))$ 
 and $X_c$ abbreviates $p(f(a)),q(a,c),q(a,f(a))$.
The above example shows the essential difficulty of the elimination
of a Skolem-axiom from a proof in CL. Observe that, in $\set{p(f(a))}$,
the Skolem-function may show up in a term before the
Skolem-axiom has been applied. Second, the tree may branch,
in this case by $q(a,a_p)\vee q(a,c)$, and there may be
\emph{different applications} of the Skolem-axiom in each branch:
the Skolem-term $f(a)$ replaces $b$ in one branch and $c$ in the other.
One is tempted to replace $f(a)$ by $b$ in the left branch
and by $c$ in the right, while at the same time eliminating
the respective applications of the Skolem-axiom. To get a closed
proof, one then needs $p(a)$ in the left, and $p(a_p)$ in the right branch.
This is in fact possible, since $p(f(a))$ has been obtained
by instantiating $p(x)$ in the first line from below.
By instantiating $p(x)$ one can equally well obtain $p(a)$ and $p(a_p)$.
This would lead to the following proof not using the Skolem-axiom:

\[
\infer{\emptyset}
{
\infer{\set{p(a)}}
 {
\infer{\set{p(a),p(a_p)}}
  {
%left
\infer{\set{p(a),p(a_p),q(a,a_p)}}
   {
\infer{\set{p(a),p(a_p),q(a,a_p),q(b,a_p)}}
    {
       \set{p(a),p(a_p),q(a,a_p),q(b,a_p),r}
    }
   }
&
%right
\infer{\set{p(a),p(a_p),q(a,c)}}
   {
\infer{\set{p(a),p(a_p),q(a,c),q(c,c)}}
    {
       \set{p(a),p(a_p),q(a,c),q(c,c),r}
    }
   }
  }
 }
}
\]

How to generalize from this example?
Let us first consider a proof that doesn't branch.


Write $\neg\phi$ as $\phi\ra\bot$: each of $\wedge,\vee,\ra,\forall,\exists$
can be the main connective/quantifier of a coherent formula.


